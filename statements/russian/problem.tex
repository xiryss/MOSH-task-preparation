\begin{problem}{Разбиение на команды}{стандартный ввод}{стандартный вывод}{15 секунд}{256 мегабайт}

Тренер Игорь очень любит готовить школьников к командным олимпиадам по программированию. 

Всего к нему в кружок ходит $N$($N$ делится на $3$) школьников, и, так как Игорь уже довольно опытный, для каждой пары он знает "коэффициент сыгранности" $C_{a,b}$ этих двух школьников. По его наблюдениям, если в одной команде находятся ученики под номерами $a,b,c$, успех их команды можно будет выразить как $C_{a,b} \cdot C_{b,c} \cdot C_{a,c}$. Игорю будет приятно, если как можно больше обучающихся успешно выступят на предстоящей Межгалактической Командной Олимпиаде Школьников по Программированию(МКОШП), и он хочет максимизировать суммарный успех составленных команд.

Иными словами, Игорь хочет максимизировать сумму $C_{a_i, b_i} \cdot C_{b_i,c_i} \cdot C_{a_i, c_i}$ по всем $i$ от $1$ до $\frac{N}{3}$, где $a_i,b_i,c_i$ - номера учеников $i$-той команде.
Его кружок очень популярен, и он не справляется с тем, чтобы оптимально решить эту задачу, так помогите же ему!

\InputFile
В первой строке вводится количество учеников $N$($1\leq N \leq 700$, $N$ делится на $3$).
В последующих $N$ строках вводятся по $N$ чисел - в $j$-ом столбце $i$-ой строки находится коэффициент сыгранности учеников под номерами $i$ и $j$.


\OutputFile
Выведите оптимальное разбиение на команды в виде $\frac{N}{3}$ строк, в каждой из которых находится по 3 целых числа - номера учеников, принадлежащих этой команде.

\Scoring
Ваше решение будет оцениваться относительно решения жюри. Если ваше решение имеет значение оптимизируемой функции $a$, а решение жюри - $b$, то вы получаете за тест $10 \cdot (\frac{a}{b})^3$ баллов. Если же $a > b$, вы получите 10 баллов за этот тест.


\end{problem}

